\documentclass[12pt]{article}
\usepackage[utf8]{inputenc}
\usepackage{tikz}
\usepackage{tkz-euclide}
\usepackage{pgfplots}
\usepackage{booktabs}
\usepackage[margin=1.1in]{geometry}
\usetkzobj{all}
\usetikzlibrary{arrows, shapes.gates.logic.US, calc}
\usepackage{minted}
\usepackage{titlesec}
\usepackage{arydshln}
\usepackage{mathtools}
\newcommand{\sectionbreak}{\clearpage}
\newcommand\inv[1]{#1\raisebox{1.15ex}{$\scriptscriptstyle-\!1$}}


\begin{document}
\begin{titlepage}
\begin{figure}[t]
    \centering\includegraphics[width=0.15\textwidth]{logo}
\end{figure}
\begin{center}
    \textsc{ \LARGE{University of Sussex \\}}
	\textsc{School of Engineering and Informatics\\}
	\vspace{12mm}
	\fontsize{6mm}{7mm}\textnormal{Mathematics for Computing 2\\}
	\fontsize{10mm}{7mm}\selectfont
	\vspace{2mm}
    \textup{Assignment 1}\\
\end{center}

\vspace{25mm}

\centering{\large{Candidate number: 215800}}

\end{titlepage}
\tableofcontents

\section{Logic and Karnaugh maps}
\subsection{Truth table}

Here is the truth table.
\begin{table}[h]
\begin{tabular}{|l|l|l|l|l|l|l|l|l|l|l|}
\hline
A & B & C & D & Gate 1 & Gate 2 & Gate 3 & Gate 4 & Gate 5 & Gate 6 & OUT \\ \hline
0 & 0 & 0 & 0 & 1 & 0 & 0 & 0 & 1 & 0 & 1 \\ \hline
0 & 0 & 0 & 1 & 1 & 0 & 0 & 0 & 1 & 0 & 1 \\ \hline
0 & 0 & 1 & 0 & 0 & 1 & 0 & 0 & 1 & 0 & 1 \\ \hline
0 & 0 & 1 & 1 & 0 & 1 & 0 & 0 & 1 & 0 & 1 \\ \hline
0 & 1 & 0 & 0 & 1 & 0 & 0 & 0 & 1 & 0 & 1 \\ \hline
0 & 1 & 0 & 1 & 1 & 0 & 0 & 0 & 1 & 0 & 1 \\ \hline
0 & 1 & 1 & 0 & 0 & 0 & 0 & 1 & 0 & 1 & 1 \\ \hline
0 & 1 & 1 & 1 & 0 & 0 & 1 & 0 & 0 & 1 & 1 \\ \hline
1 & 0 & 0 & 0 & 0 & 0 & 0 & 0 & 0 & 0 & 0 \\ \hline
1 & 0 & 0 & 1 & 0 & 0 & 0 & 0 & 0 & 0 & 0 \\ \hline
1 & 0 & 1 & 0 & 0 & 0 & 0 & 0 & 0 & 0 & 0 \\ \hline
1 & 0 & 1 & 1 & 0 & 0 & 0 & 0 & 0 & 0 & 0 \\ \hline
1 & 1 & 0 & 0 & 0 & 0 & 0 & 0 & 0 & 0 & 0 \\ \hline
1 & 1 & 0 & 1 & 0 & 0 & 0 & 0 & 0 & 0 & 0 \\ \hline
1 & 1 & 1 & 0 & 0 & 0 & 0 & 1 & 0 & 1 & 1 \\ \hline
1 & 1 & 1 & 1 & 0 & 0 & 1 & 0 & 0 & 1 & 1 \\ \hline
\end{tabular}
\end{table}

\subsection{Karnaugh map and circuit}

\begin{table}[h]
\begin{tabular}{@{}lllll@{}}
\toprule
CD - AB & 00 & 01 & 11 & 10 \\ \midrule
00      & 0  & 0  & 1  & 1  \\
01      & 0  & 0  & 1  & 1  \\
11      & 0  & 0  & 0  & 1  \\
10      & 0  & 0  & 0  & 1  \\ \bottomrule
\end{tabular}
\end{table}

[insert from paper]

\subsection{Represent circuit using NAND gates}

see paper simon

\section{Integration}

\section{Integration and design}

\section{Further integration 1}

\section{Further integration 2}

\section{Arithmetic and geometric series}

\subsection{For the series 1, 5, 9, 13 …, find the sum of the first 11 terms}

\[a=1, d=4\]

\[S_{11} = \frac{11}{2}(2+40)=231\]

\subsection{For the series -6, -6.8, -7.6 …, find the sum of terms 6 through to 10
inclusive.}

\[a=-6, d=-0.8\]

\[S_{6...10} = S_{10}-S_4\]

\[S_4=\frac{4}{2}(-12-2.4)=-28.8\]
\[S_{10}=\frac{10}{2}(-12-2.4)=-72\]

\[S_{6...10} = -42.2\]

\subsection{For the series 2, 4, 8, 16 … , find the 11
th term, and the sum of the first 13
terms.}

\[a=2, r=2\]
\[n_{11}=2\cdot2^{10}=2048\]
\[S_{13}=\frac{2(2^{13}-1)}{2-1}=16382\]

\subsection{For the series 3, -3.6, 4.32, -5.184… , find the 10
th term and state whether
this is a converging or a diverging series.}

\[a=3, r=-1.2\]
\[n_{10}=3\cdot-1.2^9=-15.48\]

The series is diverging.

\subsection{For the series -4, 3.2, -2.56, 2.048 … , find the 8
th term and the sum of an
infinite number of terms of this series.}

\[a=-4, r=-0.8\]
\[n_=-4\cdot-0.8^7=0.838\]
\[S_{\infty}=\frac{-4}{1--0.8}=2.\overline{2}\]

\section{Representing negative numbers in binary}

\subsection{Convert to two's complement binary 12-bit form}
\subsubsection{$-781$}
\[+781_2=001100001101\]

Invert it
\[110011110010\]

Add 1
\[-781_2=110011110011\]

\subsubsection{$-56$}
\[+56_2=000000111000\]

Invert it
\[111111000111\]

Add 1
\[-56_2=111111001000\]

\subsubsection{$-102$}
\[+102_2=000001100110\]

Invert it
\[111110011001\]

Add 1
\[-102_2=111110011010\]

\subsection{Perform the following calculations in binary}
\subsubsection{$78-671$}
\[78=000001001110\]
\[-671=110101100001\]
\[78+-671=110110101111\]

\subsubsection{$-376-78$}
\[-376=111010001000\]
\[-78=111110110010\]
\[-376+-78=111000111010\]

\section{Representing floating point numbers in binary}

\[27.764 = 11011.1011111100\]

Normalize
\[1.10111011111100\]

Gather information
sign = $0$
exponent = $4$
biased exponent = $131$
mantissa = $10111011111100$

\section{Set theory}

\section{Discrete probability}

\end{document}
